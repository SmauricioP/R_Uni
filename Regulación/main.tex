\documentclass{article}
\usepackage[utf8]{inputenc}
\usepackage{amsmath}
\usepackage{mathtools}

\usepackage[margin=1.15 in]{geometry}

\DeclareMathOperator*{\argmax}{arg\,max}

\setlength{\parindent}{0cm}
\setlength{\parskip}{0.15cm}

\begin{document}

\section*{Pregunta 1}

Se maximiza respecto a los precios debido a que, por definición, los precios Ramsey \textbf{maximizan el beneficio social}.
\begin{align*}
    \max_{p_1,p_2} \ &W(p_1,p_2) = \int_{0}^{f(p_1)} \left(p_1 - \text{Cmg}_1(q)\right)\text{d}q + \int_{0}^{f(p_2)} \left(p_2 - \text{Cmg}_2(q)\right)\text{d}q \\
    &\text{sujeto a} \ \  p_1f(p_1) + p_2f(p_2) - \text{CT}(f(p_1),f(p_2)) = 0
\end{align*}

Se arma el lagrangian y se plantea la condición de primer orden.
\begin{align*}
    \mathcal{L}(p_1,p_2) =& \int_{0}^{f(p_1)} \left(p_1 - \text{Cmg}_1(q)\right)\text{d}q + \int_{0}^{f(p_2)} \left(p_2 - \text{Cmg}_2(q)\right)\text{d}q \\ 
    & + \lambda \left( p_1f(p_1) +p_2f(p_2) - \text{CT}\left(f(p_1),f(p_2)\right) \right)
\end{align*}

Se plantean las condiciones de primer orden.
\begin{align*}
    \frac{\partial \mathcal{L}}{\partial p_1} &= 0 \to \left(p_1 - \text{Cmg}_1(f(p_1))\right)f'(p_1) + \lambda \left( f(p_1) + p_1f'(p_1) - \text{Cmg}_1f'(p_1)\right) = 0\\ 
    \frac{\partial \mathcal{L}}{\partial p_2} &= 0 \to \left(p_2 - \text{Cmg}_2(f(p_2))\right)f'(p_2) + \lambda \left( f(p_2) + p_2f'(p_2) - \text{Cmg}_2f'(p_2)\right) = 0
\end{align*}
Agrupando convenientemente
\begin{align*}
    (1+\lambda)\left(p_1 - \text{Cmg}_1(f(p_1))\right)f'(p_1) = -\lambda f(p_1) \to \frac{p_1 - \text{Cmg}_1}{p_1} =- \frac{\lambda}{1+\lambda}\frac{f(p_1)}{p_1f'(p_1)} \\ 
    (1+\lambda)\left(p_2 - \text{Cmg}_2(f(p_2))\right)f'(p_2) = -\lambda f(p_2) \to \frac{p_2 - \text{Cmg}_2}{p_2} =- \frac{\lambda}{1+\lambda}\frac{f(p_2)}{p_2f'(p_2)}
\end{align*}

Recordando que las elasticidades siempre son negativas, asumiendo demandas sin problemas, se pueden escribir como :

\begin{equation*}
    \varepsilon_i = p_i\frac{f'(p_i)}{f(p_i)}
\end{equation*}
Las condiciones se pueden escribir de la siguiente manera:

\begin{align}
    \frac{p_1 - \text{Cmg}_1}{p_1} = \frac{\lambda}{1+\lambda}\frac{1}{\vert \varepsilon_1 \vert} \tag{Opt 1}\\ 
    \frac{p_2 - \text{Cmg}_2}{p_2} = \frac{\lambda}{1+\lambda}\frac{1}{\vert \varepsilon_2 \vert} \tag{Opt 2}
\end{align}

\newpage

\section*{Pregunta 2}
En esta sección, se analizan las características de los precios Ramsey.
\subsection*{Los precios Ramsey \textit{son} óptimos}
Por definición, los precios Ramsey maximizan el bienestar social\textit{Aunque quizá sea más adecuado analizarlo en términos de pérdida de bienestar social} sujeto a la restricción de beneficios no extraordinarios. De esta manera, los precios Ramsey sí son precios óptimos.
\begin{equation*}
    \textbf{p}^* = \argmax_{p_1,\ldots,p_n} \ W(\textbf{p})
\end{equation*}
Donde una posible definición de la función de bienestar\footnote{De hecho, esta definición es la que se utiliza en la clase de regulación.} es la suma de los excedentes del consumidor en cada mercado pertinente.
\begin{equation*}
    W(\textbf{p}) = \sum_{k = 1}^n \int\limits_0^{q_k(p_k)}\left(p_k - \frac{\partial C(\textbf{q}(\textbf{p}))}{\partial q_k}\right)\text{d}q
\end{equation*}
\subsection*{Los precios Ramsey \textit{no son} justos}
La clave para determinar si los precios Ramsey son justos o no radica sobre el análisis de la \textit{elasticidad de la demanda}. Analizando la condición de primer orden para la maximización de los beneficios del monopolista\footnote{No es dificil ver que, como las demandas de los bienes son independientes entre sí, las condiciones halladas para dos bienes puede ser fácilmente generalizada}:
\begin{equation*}
    \frac{p_i - \text{Cmg}_i}{p_i} = \frac{\lambda}{1+\lambda}\frac{1}{\vert \varepsilon_i \vert} 
\end{equation*}
Se sabe que mientras menos sustitutos tenga un bien, más inelástico es este bien. Un bien con pocos sustitutos es muy probablemente un bien que se produzca poco y que sea importante para la sociedad\footnote{Quizá haya otra explicación. Por lo que he visto, el profesor lo explica por la parte de los ingresos.}. Sin embargo, ¿qué nos dice la condición de optimalidad? \textbf{Un bien muy inelástico tiene una disociación considerable entre el precio y el costo marginal}, lo que no permitirá que todos los demandantes accedan a este bien, especialmente los de menores ingresos. Por lo tanto, los precios ramsey \textbf{no son justos}.

\subsection*{Los precios Ramsey \textit{son} libres de subsidios cruzados}
Para garantizar que los precios Ramsey son libres de subsidios cruzados, se deben cumplir las siguientes desigualdades para toda las cantidades:
\begin{equation*}
    C(\mathbf{q}) - C(\mathbf{q}_{-i}) \leq p_iq_i \leq C(\mathbf{q}_{-i}) \ \; \forall i=1,\ldots,n
\end{equation*}
Donde se tiene que\footnote{El vector $\mathbf{e}_k$ es un vector con un uno en la posición $k$ y cero en el resto de posiciones.}:
\begin{equation*}
    \mathbf{q}_{-i} := \sum_{k\neq i}^n q_k \mathbf{e_k}
\end{equation*}
\newpage
Dado que suponemos que la función de costos es lineal y que las cantidades son positivas, se puede escribir la condición de arriba como sigue:
\begin{equation*}
    c_iq_i \leq p_iq_i \leq c_iq_i + F
\end{equation*}
La primera parte de la inecuación queda asegurada, pues se tiene que $c_i < p_i$. Para la segunda parte de la ecuación, se observa que la función de beneficios evaluada en los precios Ramsey es igual a cero, es decir:
\begin{equation*}
    \mathbf{p}^T\mathbf{q} = \mathbf{c}^T\mathbf{q} + F \to \left( \mathbf{p} - \mathbf{c}\right)^T\mathbf{q} = F
\end{equation*}
Escrito de otra manera, se puede observar lo siguiente:
\begin{equation*}
    \left(p_i-c_i\right)q_i = F - \sum_{k\neq i}^{n} \left(p_k-c_k\right)q_k \leq F
\end{equation*}
Finalmente, se concluye que $\left(p_i-c_i\right)q_i\leq F$, o, lo que es lo mismo, $p_iq_i \leq c_iq_i + F$, la segunda inecuación para determinar si los precios están libres de subsidios.
\subsection*{Los precios Ramsey \textit{son} sostenibles}
Para el análisis de la sostenibilidad del monopolio multiproducto, el profesor analiza la existencia de economías a escala globales utilizando el siguiente indicador. 
\begin{equation*}
    S_G := \frac{C(\mathbf{q})}{\mathbf{q}\cdot \Delta f(\mathbf{q})}
\end{equation*}
Donde se dice que existen economías de escala si el indicador $S_G$ es mayor a uno. Bajo el supuesto que los costos son lineales, se tiene lo siguiente:
\begin{equation*}
    S_G = \frac{c_1q_1 + c_2q_2 + F}{c_1q_1 + c_2q_2}
\end{equation*}
Se puede ver claramente que este indicador siempre será mayor a uno para todos los valores de $q_i$ y del costo marginal $c_i$. 

Más aún, en el caso general, donde $C(\mathbf{q}) = \mathbf{c}^T\mathbf{q} + F$, se observa que:

\begin{equation*}
    S_G = \frac{\mathbf{c}^T\mathbf{q} + F}{\mathbf{c}^T\mathbf{q}} > \frac{\mathbf{c}^T\mathbf{q}}{\mathbf{c}^T\mathbf{q}} = 1
\end{equation*}
Por lo tanto, los precios Ramsey \textbf{son sostenibles}. 
\newpage

\section*{Pregunta 3}

Se analizará la solución numérica de los precios de Ramsey, partiendo de las condiciones de optimalidad halladas en la primera sección. Para esto, se asumirán las siguientes formas funcionales:
\begin{align*}
    \text{CT}(q_1,q_2) &= c_1q_1 + c_2q_2 + F \\
    q_1 &= a - bp_1\\
    q_2 &= c - dp_2
\end{align*}
Si despejamos ambas condiciones de optimalidad e igualamos, se obtiene la siguiente relación:
\begin{equation}
   \frac{p_1 - c_1}{p_1} \varepsilon_1= \frac{p_2 - c_2}{p_2} \varepsilon_2 \tag{Ramsey} 
\end{equation}
Calculando las elasticidades precio de la demanda de cada bien:
\begin{equation*}
    \varepsilon_1 = -b \; \frac{p_1}{q_1} \ \, , \ \, \varepsilon_2 = -d \; \frac{p_2}{q_2}
\end{equation*}
Reemplazando las elasticidades en la condición hallada:
\begin{align*}
   \frac{p_1 - c_1}{p_1} -b  \frac{p_1}{q_1} &= \frac{p_2 - c_2}{p_2}  -d \frac{p_2}{q_2} \\
   b\left(p_1 - c_1\right)q_2 &= d\left(p_2 - c_2\right)q_1
\end{align*}
Reemplazando las funciones de demanda en la relación superior para hallar una relación entre las cantidades:
\begin{align*}
    (a - bc_1 - q_1)q_2 &= (c - dc_2 - q_2)q_1 \\[4pt]
    (a - bc_1)q_2 &= (c-dc_2)q_1 \\[0pt]
    q_2 &= \frac{c-dc_2}{a - bc_1}q_1 \label{eq1}\tag{1}
\end{align*}
Por otra parte, teniendo en cuenta la restricción de beneficios no extraordinarios:
\begin{align*}
    (p_1-c_1)q_1 + (p_2-c_2)q_2 = F
\end{align*}
De manera similar, se reemplazan los $p_i$ para que todo quede en función de cantidades\footnote{Como la función de demanda es $q_1 = a - bp_1$, se puede despejar $p_1$ y restar por $bc_1$ a ambos lados para obtener $b(p_1 - c_1) = a - bc_1 - q_1$. Se procede de manera similar para despejar $p_2 - c_2$.}.
\begin{equation}
    \frac{1}{b}(a - bc_1 - q_1)q_1 + \frac{1}{d}(c - dc_2 - q_2)q_2 = F \tag{2}\label{eq2}
\end{equation}
Es acá donde empleamos la relación \eqref{eq1} en \eqref{eq2} para obtener la siguiente expresión:
\begin{equation*}
    \frac{1}{b}(a - bc_1 - q_1)q_1 + \frac{1}{d}\left(c - dc_2 - \frac{c-dc_2}{a - bc_1}q_1\right)\frac{c-dc_2}{a - bc_1}q_1 = F 
\end{equation*}
Donde, simplificando los términos, se obtiene el siguiente polinomio de segundo grado:
\begin{equation}
    -\left( \frac{1}{d} + \frac{1}{d}\left( \frac{c-dc_2}{a-bc_1} \right)^2\right)q_1^2 + \left( \frac{a-bc_1}{b} + \frac{c-dc_2}{d}\left( \frac{c-dc_2}{a-bc_1} \right)\right)q_1 - F = 0 \label{eq3} \tag{3}
\end{equation}
\newpage
Afortunadamente, la solución de este tipo de ecuaciones es muy sencilla. Probablemente las raíces sean reales, diferentes y positivas. La solución elegida será la mayor, debido a que este valor maximiza la función objetivo, $W$.

Una vez hallado $q_1^*$, se pueden deducir fácilmente el resto de valores, ya que para hallar $q_2^*$ solo basta utilizar la relación \eqref{eq1} y para hallar los precios solo basta reemplazar en la función de demanda de cada bien.

Finalmente, la pérdida de eficiencia social puede ser hallada fácilmente como el área de los triángulos de pérdida social en cada mercado, donde se necesitará hallar primero las cantidades competitivas:
\begin{align*}
    \text{PES}_1 &= \frac{1}{2} \left(p_1^* - c_1\right)\left(q_1^{\text{cp}} - q_1^*\right) \\
    \text{PES}_2 &= \frac{1}{2} \left(p_2^* - c_2\right)\left(q_2^{\text{cp}} - q_2^*\right)
\end{align*}
Donde se tiene que $q_1^{\text{cp}} = a - bc_1$ y $q_2^{\text{cp}} = c - dc_2$, los equilibrios competitivos de cada mercado.
\end{document}
